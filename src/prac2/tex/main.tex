\begin{document}

\maketitle % Insert the title, author and date
This is the second of two pracs on seismic data processing.  In this prac we will be processing the dataset generated in prac 1.  
\par~\\
If you do not have the dataset from prac 1 is it available for download at \textbf{http://goo.gl/YuZgE9}
\par~\\
As with prac 1 code templates have been provided. The prac can be downloaded from  \\ \textbf{https://github.com/stuliveshere/Seismic-Processing-Prac2}

\section{Introduction}
Seismic data processing involves taking raw seismic data and generating an image which can be related to the geology targeted by the seismic survey.  Seismic processing usually involves a number of steps, including
\begin{itemize}
\item velocity analysis
\item geometric corrections (normal moveout,  statics, migration)
\item amplitude corrections (amplitude recovery, AGC)
\item noise attenuation
\item CMP stacking
\item deconvolution
\end{itemize}
Although not necessarily in this order.
\par~\\
For this prac we will use the following processing sequence:
\begin{enumerate}
\item load and QC (quality control) data
\item sort into CDP gathers
\item amplitude recovery
\item move out with constant velocity
\item field stack (aka brute stack)
\item first velocity analysis
\item pre-stack noise attenuation
\item first velocity stack
\item post-stack noise attenuation
\end{enumerate}
This is somewhat like a processing sequence used in a commercial processing house.  Note the iterative nature of the sequence - this is a common trait of seismic processing.
\par~\\
The final image quality is influenced by the accuracy of the seismic velocities and the parameters selected.  inaccurate velocities can lead to imperfect moveout which can affect stack quality.  poor selection of processing parameters can harm the data - for example a harsh noise attenuation routine might also attenuate the signal as well as the noise. finding optimum parameters usually involves significant amounts of testing.
\par~\\
In this prac you will be picking your own velocities and selecting your own processing parameters.  You will be expected to test a range of parameters and discuss why you chose those parameters in your report.  
\subsection*{Exercise 1: Load data and perform QC}
In this exercise we will create an initialization routine which will load our dataset into numpy, and set up some initial parameters. We will then have a look at some raw shot gathers to ensure the data is loaded correctly (QC).
\subsection*{Exercise 2: Sort into CMPs}
CMP gathers collect all traces which pass through the same CMP into a single gather.  CMP gathers are generally used for velocity analysis and stacking. The number of traces in a gather is referred to as the fold.  
\par~\\
Sort the dataset into CMP gathers and perform QC.  Compare the CMP gathers from this exercise to the shot gathers from exercise 1 and discuss why they look different.
\par~\\
Extract one CDP gather for later testing.  

\subsection*{Exercise 3: Amplitude Recovery}
We have already seen in the raw gathers that deeper signals are weaker and much more difficult to see. In the last prac we compensated for this using an AGC.  An AGC normalizes the amplitude in a sliding window over every trace, based upon the amplitudes in the window.  Sometimes in seismic processing we use an AGC prior to stack, but sometimes we want to preserve the relative amplitude of samples between traces.  We call this amplitude recovery.
\par~\\
In theory we could correct directly for spherical divergence, but it is much more challenging to correct for the remaining sources of attenuation.  Thus a generic, tunable formula is generally used, where testing is used to fine tune the result until it "looks right".  One such formula is
\[ R = e^{\gamma t}\],

where
\begin{itemize}
\item $R$ is the amplitude correction
\item $\gamma$ is the tunable parameter
\item $t$ time
\end{itemize}

Write a python function which implements this formula, and apply to your test gather.  Test various values of $\gamma$ such that the amplitude of the bottom reflector in the gather is closer in amplitude to the top reflectors. Show your gather before and after amplitude recovery. 
\par~\\
\emph{Hint: $e^x$ is called by the numpy function np.exp().}
\par~\\

\subsection*{Exercise 4: Normal moveout correction}
Notice in the raw CMP gathers the reflectors have a hyperbolic shape.  In order to stack a CMP the reflectors need to be flat.  The reason for the hyperbolic shape is normal moveout, and flattening reflectors is generally referred to as normal moveout correction.  For a flat, horizontal reflector, the travel-time equation is
\[ {t_x}^2 = {t_0}^2 + \frac{x^2}{v^2}\]
where
\begin{itemize}
\item $t_x$ is the travel time at offset x
\item $t_0$ is the travel-time at offset 0
\item $x$ is the offset
\item $v$ is the velocity
\end{itemize}
Write a python function which implements the normal moveout correction. That is, re-arrange the above equation so that we can calculate the zero offset travel-time $t_0$.
\par~\\
The function which applies your move out correction to the dataset has been supplied. Using your test CMP gather, try to find a single velocity (aka constant velocity) which flattens the reflectors the most. Show that gather before and after moveout correction.  Discuss the issues associated with normal moveout with a constant velocity.
\par~\\
Note the code relating to "stretch".  The amount of moveout correction is not linear, and thus shallow data tends to get stretched. Experiment with stretch values on your test CMP gather. Show a gather with and without stretch.  Discuss the potential impact of stretched data on a stack.

\subsection*{Exercise 5: Field Stack}
Stacking involves taking all of the traces in a single CMP gather and summing together into once trace.  The amplitude of the trace is then normalized by the fold of the gather.  Thus stacking can also be considered as finding the average trace at a given CMP location.  When all CMP gathers are stacked, the resulting image has one trace at each CMP location.  This is generally referred to as a stack or a stacked section.
\par~\\
Write a function which will stack any gather given to it.  Note this can be done in one line. 
\par~\\
\emph{Hint: $mean = \bar{x} = \frac{x_1 + x_2 + \ldots + x_n}{n}$}
\par~\\
\emph{An even bigger hint: think about the axis you want to use numpy.mean() along}
\par~\\
Once you have stacking working, you are ready to do a field stack.  Initialise the entire seismic volume, apply TAR, NMO, and then stack.  Ask the tutors to check your code before you run it.  This will take a few minutes to run - the NMO is not particularly efficient. Try it with and without pre-stack AGC. What are some of the problems with the stack? What is the large feature above the coals?
  

\subsection*{Exercise 6: First Velocity Analysis}
In exercise 4 we moved out the test CDP with a constant velocity.  A more realistic scenario would be to pick a number of velocities, one for each layer. One method of determining the optimal stacking velocities is known as semblence analysis.  Essentially, a given CDP is moved out and stacked with a range of velocities. The best stacking velocity will result in the highest amplitude stack for a given reflector.  By visualising all the velocities at once, you can see exactly where the maximum stacking energy occurs.
\par~\\
A simplistic semblence routine has been given to you in exercise 6. Run exercise 6 on your test CDP. Pick velocities and times for the 3 layers, and make a note of these for later.  What is the large high energy event which can also be seen in the semblence plot?

\subsection*{Exercise 7: Noise attenuation - Top Mute}
As seen in the stack in Exercise 5, and in the semblence plot in exercise 6, parts of the refractor are being treated as reflectors and are being stacked into our dataset.  There are a number of different ways we could attenuate or even entirely remove the refractor.  The refractor often has lower frequency than our reflectors, so a frequency filter might knock back the high amplitude low frequency components. But in this case the refractor is very linear and can be targetted by a mute window.
\par~\\
We could calculate the dip of the refractor, come up with a geometric formula, calculate the time to refractor at a given geophone, and then apply a mute, but an easier method is to use what is called a linear move-out (LMO).  An LMO is similiar in concept to an NMO, but instead of flattening a hyperbolic reflector it flattens a linear event.  Because the event is linear, the traveltime equation is our old friend the velocity equation
\[ v = \frac{x}{t} \]
where
\begin{itemize}
\item v = velocity
\item x = (absolute) offset
\item t = time
\end{itemize}
Derive an equation which will apply a linear moveout \emph{correction} to a dataset.  
\par~\\
\emph{Hint: we want to remove the dip.  Remove kind of sounds like we want to subtract something}
\par~\\
The function which applies your linear moveout correction to a dataset has been supplied. Experiment with a couple of velocities to make sure it's working ok.  Calculate the exact velocity which will flatten the refractor.
\emph{Hint:$\Delta v = \Delta x/\Delta t$}
\par~\\
Once you have your refractor nice and flat, we will need to mute it out.  Find the time of the now flat refractor, and overwrite the refractor with zeros. 
\par~\\
You will now need to reverse the LMO so the CDP gather looks normal again.  Show your CDP gather before LMO, after apply LMO, after muting, and after removing the LMO. Describe how you calculated the LMO velocity.

\subsection*{Exercise 8 : First Velocity stack}
You now have all the components to generate a minimally processed stack with picked velocities.  This stack is sometimes referred to as the "first velocity stack".  You will need to
\begin{enumerate}
\item initialise the full volume
\item apply the amplitude recovery
\item apply LMO to mute refractor
\item apply NMO to flatten refractors
\item apply AGC so we can see it better
\item stack volume
\item view
\end{enumerate}
\par~\\
Make sure you have used all your parameters correctly from previous exercises - you dont want to have to re-run your code.  The tutors will check your code before you start.  You will also want to output the stack as a file, so that you dont have to re-run your code for the next exercise.
\par~\\
Compare your first velocity stack to the brute stack we generated earlier. What are the main differences?  Discuss why we applied the LMO before the NMO.

\subsection*{Exercise 9: Post-stack noise attenuation}
Consider what happens when you run a smoothing window along a trace.  Events which are shorter than the smoothing window will tend to be smoothed out - events which are longer than the smoothing window will tend to be left along. Now consider what happens when you run a smoothing window along a time slice, between traces.  Events shorter than our smoothing window, for example random noise, will tend to be attenuated, and events longer than our window will tend to be left alone. This sounds like a good thing, so lets try it.
\par~\\
The process is actually a form of trace mixing, although the version we are going to use today is slightly different, because we are going to use the power of numpy. This code has been supplied to you in exercise 9. Experiment with mixing window sizes to find one that works well.  What are some of the downsides of trace mixing?
\par~\\
\emph{Extra marks: perform the trace mixing pre-stack.  What are the pros and cons of a pre-stack trace mix?}


\newpage
\lstinputlisting{../exercise1.py}
\lstinputlisting{../exercise2.py}
\lstinputlisting{../exercise3.py}
\lstinputlisting{../exercise4.py}
\lstinputlisting{../exercise5.py}
\lstinputlisting{../exercise6.py}
\lstinputlisting{../exercise7.py}
\lstinputlisting{../exercise8.py}
\lstinputlisting{../exercise9.py}


\end{document}